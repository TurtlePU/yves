\documentclass[manuscript,screen,review]{acmart}
\settopmatter{printacmref=false} % remove when accepted
\authorsaddresses{} % remove when accepted

\usepackage{color,listings,mathpartir}
\def\lstlanguagefiles{lstlean.tex}
\definecolor{keywordcolor}{rgb}{0.7, 0.1, 0.1}   % red
\definecolor{tacticcolor}{rgb}{0.0, 0.1, 0.6}    % blue
\definecolor{symbolcolor}{rgb}{0.0, 0.1, 0.6}    % blue

\acmConference[ICFP'25]
  {International Conference on Functional Programming}
  {October 12--18, 2025}{Singapore}

\begin{document}

\title{Functional Pearl: DIY Elaborator for Internal Solvers
  Using Time Travel, Transformers, Free Scoped Monads and Term Reconstruction}

%\begin{abstract}
%  In this talk, we present a minimal lambda calculus with dependent types and
%  bidirectional type checking which would allow internal implementation of
%  SMT solvers, without external script languages or metaprogramming features.
%  For demonstration, we implement a solver for an equational theory of monoids.
%\end{abstract}
%
%\keywords{dependent types, bidirectional typing, partial inversion}

\maketitle

\section{Introduction} \label{intro}

Being able to formally verify a complex mathematical theorem or to prove that a
critical piece of software runs correctly is always a delight. Due to the formal
nature of the task, however, this requires lots of technical lemmas to be
proven, which are nigh trivial for a human mind but which computer might
struggle to solve automatically. Consider the following lemma from Lean's
\texttt{mathlib4} (in \texttt{Algebra.Group.Basic}):

\begin{lstlisting}[language=Lean]
variable [CommSemigroup G]
theorem mul_right_comm (a b c : G) : a * b * c = a * c * b := by
  rw [mul_assoc, mul_comm b, mul_assoc]
\end{lstlisting}

This code snippet states and proves that, in any commutative semigroup $G$,
given three elements $a, b, c \in G$, it is true that $a\cdot b\cdot c
=a\cdot c\cdot b$. To mathematically educated person, the correctness of the
lemma itself is obvious; however, the one who wrote the proof needed three steps
of rewriting the type (by associativity of multiplication, by commutativity and
then by associativity again) to convince the typechecker.

It would surely be nice to avoid this chore! To aid us all, proof assistant
designers usually provide the means to automate the generation of proofs for
trivial lemmata. One viable approach is to use SMT solvers which, given a
statement formulated in one of the more restricted logical theories, decide
whether the statement is true and output the proof. Usually, such solvers are
written in another programming language or use metaprogramming facilities.

In this paper, we contribute in the following ways:
\begin{itemize}
  \item In Section \ref{solver} we sketch a technique which enables a user of a
    proof assistant to write solvers for decidable theories inside proof
    assistant and immediately use them in this same system without any macros.
  \item To back up our claims, in Section \ref{typing} we describe a suitable
    minimal type system and in Section \ref{impl} we provide a proof-of-concept
    elaboration algorithm written in Haskell.
  \item By composing well-known solutions from pure functional programming world
    and making use of Haskell's laziness, in Section \ref{impl} we arrive at the
    description of an elaboration algorithm as a single-pass AST traversal.
\end{itemize}

\section{Sketch of a solver} \label{solver}

Essentially, our technique relies on two tricks:
\begin{enumerate}
  \item Choosing a nice type for a solver function written in a prover language;
  \item Structuring typing rules in such a way so that most of the arguments to
    said function can be inferred (this task is explored in more detail in
    Section \ref{typing}).
\end{enumerate}

The task seems circular: we need to specify the type for a solver function, so
we need to fix the type system; but to specify typing rules, we need to know the
type of the solver to design against. Lucky for us, the type of a solver is
expressible in any type system featuring the following:

\begin{enumerate}
  \item The empty type $\bot$;
  \item The disjoint union (or a sum type) $A+B$;
  \item The type of dependent functions $(x:A)\to B(x)$ (in the case where $B$
    does not depend on $x$, we write $A\to B$);
  \item The type of dependent pairs $(x:A)\times B(x)$ (in the case where $B$
    does not depend on $x$, we write $A\times B$);
  \item Ability to form inductive datatypes, including but not limited to:
    \begin{itemize}
      \item The type of natural numbers $\mathbb{N}$;
      \item For each $n:\mathbb{N}$, the type of natural numbers less than $n$,
        denoted $\underline{n}$.
    \end{itemize}
  \item The type of propositions $\mathbf{prop}$;
  \item The type of small types $\mathbf{type}$;
  \item The propositional equality type $a=_A b$.
\end{enumerate}

As an example, let's consider the construction of a solver type for equational
theory of commutative (multiplicative) semigroups, that is, a theory where:
\begin{itemize}
  \item We can ask questions about equalities of finite expressions written
    using infix multiplication $(\cdot)$ and finite number of variables
    $x_1,\ldots,x_n$.
  \item Two expressions are considered equal if we can get one from another
    applying a finite sequence of rewrites of two kinds:
    $a\cdot(b\cdot c) = (a\cdot b)\cdot c$ and $a\cdot b=b\cdot a$.
\end{itemize}

From user's point of view, such solvers are directly applicable to automatically
prove statements as the one used in the example from Section \ref{intro},
$(a\cdot b)\cdot c = (a\cdot c)\cdot b$.

Then, in a type system having features outlined above, the following terms can
be defined:

\begin{enumerate}
  \item A type of \textit{expressions} $E:(n:\mathbb{N})\to\mathbf{type}$ in our
    theory is a type of binary trees where nodes denote multiplication and
    leaves store variable indices taken from $\underline{n}$;
  \item A \textit{formula} $F$ with $n$ variables is a pair of expressions
    (left-hand side of an equation, right-hand side):
    \[F(n:\mathbb{N})\coloneq(E\;n)\times(E\;n);\]
  \item A \textit{model} $\mathcal{M}$ with carrier $X$ is a dictionary of
    functions and properties which turn a type $X$ into a semigroup:
    \[\mathcal{M}(X:\mathbf{type})
      \coloneq((\cdot):X\to X\to X)
      \times((a\;b\;c:X)\to a\cdot(b\cdot c)=(a\cdot b)\cdot c)
      \times((a\;b:X)\to a\cdot b=b\cdot a);\]
  \item An \textit{embedding function} $P$ which interprets formulas of a theory
    as internal propositions:
    \[
      P:(n:\mathbb{N})\to(X:\mathbf{type})\to(M:\mathcal{M}\;X)
      \to(\mathcal{C}:\underline{n}\to X)\to F\;n\to\mathbf{prop};
    \]
  \item An \textit{internal proof} $\mathrm{Prf}$ of a formula $\phi$ is a proof
    that it holds universally, over all models and interpretations:
    \[
      \mathrm{Prf}(n:\mathbb{N})(\phi:F\;n)\coloneq
      (X:\mathbf{type})\to(M:\mathcal{M}\;X)\to(\mathcal{C}:\underline{n}\to X)
      \to P\;n\;X\;M\;\mathcal{C}\;\phi
    \]
  \item A \textit{decision function} $\mathcal{D}$ which, in fact, expresses the
    logic of a solver -- here it's enough to compare ``normal forms'' of two
    expressions by counting number of occurences of each variable in each
    expression:
    \[
      \mathcal{D}:(n:\mathbb{N})\to(\phi:F\;n)\to
      (\mathrm{Prf}\;n\;\phi+(\mathrm{Prf}\;n\;\phi\to\bot));
    \]
  \item Finally, we would like to define the following ``wrapper''
    \begin{multline*}
      \mathbf{solver}:(n:\mathbb{N})\to(\phi:F\;n)
      \to(\pi:\mathrm{Prf}\;n\;\phi)
      \to(\mathrm{eq}:\mathcal{D}\;n\;\phi=_{\ldots}\mathrm{inl}\;\pi)
      \to\\\to(X:\mathbf{type})\to(M:\mathcal{M}\;X)
      \to(\mathcal{C}:\underline{n}\to X)
      \to P\;n\;X\;M\;\mathcal{C}\;\phi.
    \end{multline*}
    The result of this function can be computed as simply as
    $\pi\;X\;M\;\mathcal{C}$; the secret sauce is the type. It is constructed in
    such a way that arguments can be provided to the $\mathbf{solver}$ iff
    $\mathcal{D}$ states that a formula is, in fact, provable.
\end{enumerate}

In order for this framework to be practical, however, all of the wrapper's
arguments (except, maybe, for $\mathrm{eq}$, $M$ and $\mathcal{C}$) have to be
inferrable from resulting type. For example, lemma from Section \ref{intro}
would then be provable as simply as
\[
  \mathbf{solver}\;\_\;\_\;\_\;(\mathrm{refl}\;\_\;\_)\;\_\;M\;[a,b,c]
  :a\cdot b\cdot c=_{\_}a\cdot c\cdot b,
\]
where $\mathrm{refl}\;A\;x:x=_A x$, underscores denote terms which should be
inferred by the typechecker and $[x_1,\ldots,x_n]$ is a shorthand for a function
from $\underline{n}$ to $X$ mapping $i$ to $x_i$. For ease of presentation, we
refrain from adding further niceties to the surface syntax (such as implicit
arguments and instance resolution), but they of course should be present in any
practical system.

%To this end, we develop a special bidirectional typing system \cite{bidir} for
%Martin-L\"of type theory with implicit arguments. Inference of arguments is
%achieved via function inversion as in $\mathrm{Haskell}^{-1}$ \cite{haskell}
%(here, we invert $P$ to infer $\phi$) and via unification algorithm embedded in
%the typing relations (here, we unify $\mathcal{D} \; n \; \phi$ with
%$\mathrm{inl} \; \pi$). Note that this gives rise to general inference system
%which might have wider applications.

\section{Typing rules} \label{typing}
%Basic calculus is Martin-L\"of type theory with universes \cite{mltt}, a typed
%lambda-calculus which can be viewed as a terse notation for proofs in
%higher-order intuitionistic logic. MLTT includes \textbf{types} for formal
%description of mathematical structures (which can be viewed as sets under
%set-theoretical semantics) and \textbf{terms} which describe constructions of
%type inhabitants (under set-theoretical semantics, terms correspond to elements
%of said sets). As a logical system, MLTT is a system of natural deduction with
%rules for type formation ($\Gamma \vdash \tau \; \mathrm{type}$), term
%introduction and term elimination ($\Gamma \vdash t : \tau$), type equality
%($\Gamma \vdash \sigma \equiv \tau$), term equality
%($\Gamma \vdash t \equiv u : \tau$) and context well-formedness
%($\vdash \Gamma \; \mathrm{context}$). Most notably, MLTT includes rules
%necessary for declaration of inductive types (interpreted as well-founded sets),
%e.g. types of a) natural numbers; b) initial segments of natural number line
%$\underline{n}$; c) expressions in an algebraic signature with variables ranging
%over any given type. In addition, it contains a special mechanism to treat types
%as elements of special types $U_i$ called ``universes'' which we do not cover
%in this section. On the other hand, MLTT can be viewed as a total functional
%programming language which allows writing down functional algorithms AND prove
%their correctness in the same language. All this machinery allows us to write
%down a general schema of an internal solver for some decidable theory:



%\begin{mathpar}
%\inferrule[app]
%  {\Gamma\vdash f\;\$\;t,\Delta:T}
%  {\Gamma\vdash f\;t\;\$\;\Delta:T}
%\and\inferrule[var]
%  {x:U\in\Gamma\\\Gamma\vdash U\;@\;\Delta:T}
%  {\Gamma\vdash x\;\$\;\Delta:T}
%\and\inferrule[@-pi]
%  {\Gamma;\sigma\vdash t\Leftarrow T;\tau
%  \\\Gamma;\rho\vdash U[x\coloneq t]\;@\;\Delta:V;\sigma}
%  {\Gamma;\rho\vdash(x:T)\to U\;@\;t,\Delta:V;\tau}
%\and\inferrule[@-check]
%  {\Gamma\vdash T=U\Leftarrow\mathcal{U}}
%  {\Gamma\vdash T\;@\;\cdot\Leftarrow U}
%\and\inferrule[@-infer]{ }{\Gamma\vdash T\;@\;\cdot\Rightarrow T}
%\end{mathpar}

%\section{All elaboration rules}

%\begin{mathpar}
%\inferrule[type]{ }{\Gamma\vdash\mathcal{U}_l\Rightarrow\mathcal{U}_{l+1}}
%\and\inferrule[pi]
%  {\Gamma;\sigma\vdash T\Rightarrow\mathcal{U}_{l_T};\tau
%  \\\Gamma,x:T;\rho\vdash U\Rightarrow\mathcal{U}_{l_U};\sigma}
%  {\Gamma;\rho\vdash(x:T)\to U\Rightarrow\mathcal{U}_{l_T\lor l_U};\tau}
%\and\inferrule[abs-0]
%  {\Gamma;\sigma\vdash T\Leftarrow\mathcal{U};\tau
%  \\\Gamma,x:T;\rho\vdash f\Rightarrow U;\sigma}
%  {\Gamma;\rho\vdash\lambda(x:T).f\Rightarrow(x:T)\to U;\tau}
%\and\inferrule[pair]
%  {\Gamma\vdash f\Rightarrow T\\\Gamma\vdash s\Leftarrow U[x\coloneq f]}
%  {\Gamma\vdash(f,s:U)\Rightarrow(x:T)\times U}
%\and\inferrule[id]
%  {\Gamma\vdash t\Leftarrow T\\\Gamma\vdash u\Leftarrow T
%  \\\Gamma\vdash T\Rightarrow\mathcal{U}_l}
%  {\Gamma\vdash t=_T u\Rightarrow\mathcal{U}_l}
%\and\inferrule[app]
%  {\Gamma;\sigma\vdash f\;@\;t,\zeta\Rightarrow r;\sigma'}
%  {\Gamma;\sigma\vdash f t\;@\;\zeta\Rightarrow r;\sigma'}
%\and\inferrule[abs]
%  {x=t:T,\Gamma;\sigma\vdash f\;@\;\zeta\Rightarrow r;\sigma'}
%  {\Gamma;\sigma\vdash \lambda (x:T).f\;@\;t,\zeta\Rightarrow r;\sigma'}
%\end{mathpar}

\section{Implementation} \label{impl}

\section{Related work}

%We wish to provide an alternate point of view on solvers of the second kind:
%they are usually implemented either in supporting programming language (e.g.
%OCaml for Rocq \cite{coq}, Java for Arend \cite{arend}) or in the language of
%prover itself via metaprogramming facilities (e.g. how it's done in Lean
%\cite{lean} and Agda \cite{agda}).
%We believe our approach is beneficial in three ways:
%\begin{itemize}
%  \item Such solvers are correct by construction thanks to the prover;
%  \item Newcomers can explore solver implementation without having to learn
%    other languages;
%  \item Implementation of the prover becomes smaller and easier to audit.
%\end{itemize}

\bibliographystyle{ACM-Reference-Format}
\bibliography{bibliography}

\end{document}
